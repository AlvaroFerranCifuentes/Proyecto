\section{Introduction}

The Merriam-Webster dictionary defines a robot as ``a machine that can do the work of a person and that works automatically or is controlled by a computer". From the purely mechanical Renaissance automata, like the walking and praying monk in the Smithsonian Instute's collection\cite{automata-monk}  to the state of the art ASIMO, referenced in the following chapter, robots have always spurred the collective imagination and enthusiam of their builders.\\
	
	
From the mid-twentieth century the robot universe has vastly expanded, and they are now present in almost every field, ranging from construction\cite{robot-construction} to space\cite{robot-space} or medicine\cite{robot-medicine}. However, a new type of robot is beginning to appear: the domestic robot. This can be either purely social, limited to interacting and connecting emotionally with people\cite{robot-social} or assistive, designed to carry out tasks a person needs help with\cite{jardon2011personal}.\\

Until now, most assistive robots were limited to specific tasks, like helping with feeding or transporting people. This proyect however intends to build a general purpose assistive robot, that users can telecontrol from their phones and later on program simple routines to be repeated at a certain frequency.


\subsection{Socio-economic factors}

Developped countries' populations are ageing. With improved health systems the average life expectancies are getting higher. In Spain this situation is even more noticeable, since in a study by realized by the World Health Organization (WHO) \cite{who-life} it is shown that Spaniard women have the highest life expectancy in Europe and the second highest in the world, only behind Japan.\\

While increasing life expectancies is a victory for individuals, combined with lowering birth rates results in average population age increments. This means that there will be more elder people depending on the younger generations and not enough of the latter to help them in daily activities.
With the actual trend of continuously increasing life expectancies in Europe \cite{life-future}, the situation in those countries will be even more pronounced.\\

One solution may be to include social, domestic robots that would assist with domestic chores. The proposed solution tries to tackle this problem by doing exactly that: a low cost robot that anybody can buy or build and improve as needed.\\

The components are chosen taking into account reach and cost. The controller is an Android smartphone, since up to 66 percent of the population in Spain owns a smartphone\cite{numero-smartphone}, and out of those around 85 percent are Android\cite{numero-android}. Android phones also come in all prices, so users without a phone can buy the least expensive models and still be able to use the robot.\\

The on-board electronic brains, a computer and a microcontroller board are a Raspbery Pi and an Arduino respectively. These have been chosen for their capabilities, but especially for their price tags: the former costs 40\euro  and the second 23\euro, far below the 300\euro price line most ``cheap" computers cost.\\

Finally, the robot parts are 3D printed, since that is much more cost-efficient in small batches than mold injection, especially for an evolving product like a robot improved by the user community.


\subsection{Proposed solution}
The proposed solution is the creation of a Personal Domestic Service Droid (PD-SD), a robot with a humanoid upper body on a wheeled base that will be capable of assisting elderly people or with physical dissabilities. While the final robot should be of full human-size, to be able to lower high objects, a smaller prototype will be presented in this project. This is done to keep the focus on the software, which is identical and can be exported directly to a full-size robot when it is perfected, while creating a scale model to use while improving the programs.\\

The PD-SD is controlled from the user's Android smartphone, which will connect to the robot's wifi network to control it. The latter will also stream video from its camera to the smartphone, while obbeying the instructions given from the Android application. It has two arms with 5 degrees of freedom each to be able to grasp objects or perform simple actions such as closing doors, and a differential-drive base to move throughout space while being able to rotate in place, which may be useful to avoid getting stuck in small spaces such as corridors.\\










\subsection{Scope of the project}

This project's ultimate objective is to build a robotic assistant prototype, which may be controlled by a user from their own smartphone and which will stream a video feed from its camera so the user is able to see through the robot. This prototype is built to develop the technology that can be later used in a full-scale robot capable of assisting elderly people or with disabilities.\\

In order to achieve this result, the project is divided into the following objectives:

	\begin{enumerate}[I.]

		\item Study of different robot configurations to find the optimal choice for domestic assistive activities.\\

		\item Design of the aforementioned robot.\\

		\item Creation of the designed parts with a 3D printer.\\

		\item Assembly of the robot's body and installation of the electronic components.\\

		\item Program the microcontroller to understand the user's commands and control the actuators accordingly.\\

		\item Program the onboard computer to: 
			\begin{enumerate}[i.] 
			\item set up a wifi network to communicate with the user
			\item stream video from the attached camera 
			\item send the user's commands to the microcontroller.
			\item do all of the previous automatically when the system boots.\\
			\end{enumerate}

	
		\item Program the Android application the user will use to control the robot and receive the video feed it streams.

	\end{enumerate}


% \section{Project phases}
% [include Gantt diagram]

% \section{Components and equipment employed}
% 	The following catalogue contains all the components employed during the scope of the project.

% 	\begin{itemize}

% 		\item \textbf{3D printer}

% 	\end{itemize}