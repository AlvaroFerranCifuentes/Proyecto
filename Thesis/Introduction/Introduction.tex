\section{Introduction}




\section{Socio-economic factors}

ageing population, prevision spain, social robots, etc


Android: great market share, cheap

Arduino, raspberry: very cheap 20+35  vs 300 normal pc

3d printer vs injection- / kg vs / kg




\section{Regulatory compliance}

complies with normativa etc y normativa etc2

\section{Scope of the project}

This project's ultimate objective is to build a robotic assistant prototype, which may be controlled by a user from their own smartphone and which will stream a video feed from its camera so the user is able to see through the robot. This prototype is built to develop the technology that can be later used in a full-scale robot capable of assisting elderly people or with disabilities.\\

In order to achieve this result, the project is divided into the following objectives:

	\begin{enumerate}[I.]

		\item Study of different robot configurations to find the optimal choice for domestic assistive activities.\\

		\item Design of the aforementioned robot.\\

		\item Creation of the designed parts with a 3D printer.\\

		\item Assembly of the robot's body and installation of the electronic components.\\

		\item Program the microcontroller to understand the user's commands and control the actuators accordingly.\\

		\item Program the onboard computer to: 
			\begin{enumerate}[i.] 
			\item set up a wifi network to communicate with the user
			\item stream video from the attached camera 
			\item send the user's commands to the microcontroller.
			\item do all of the previous automatically when the system boots.\\
			\end{enumerate}

	
		\item Program the Android application the user will use to control the robot and receive the video feed it streams.

	\end{enumerate}


\section{Project phases}
[include Gantt diagram]

% \section{Components and equipment employed}
% 	The following catalogue contains all the components employed during the scope of the project.

% 	\begin{itemize}

% 		\item \textbf{3D printer}

% 	\end{itemize}